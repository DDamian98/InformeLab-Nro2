\section{Conclusion} 

\begin{itemize}


	\item Se cumplieron los pasos para atachar la base de datos y de abrir el proyecto con el contexto. Se asume que estamos en un proceso luego de haber generado la base de datos a base de First Code, por lo tanto es que contamos con estos dos recursos para a continuación hacer las pruebas.
	\item Para efectuar las distintas pruebas, nos es necesario contar con datos en el contexto del proyecto, de lo contrario no podríamos hacer las pruebas y nos arrojaría error. Además de tener problemas de inserción con la clase de Students. Fue así que se optó por desarrollar las pruebas con la clase Grades.
	\item Ejecutamos las pruebas y tenemos un Profiler conectado que está pendiente de todas las sentencias SQL que se ejecutan por detrás. Este nos permitió tener visión de lo que ocurre entre el proyecto y la base de datos, por esto es que consideramos a este elemento como parte del proceso de pruebas en una base de datos.
	\item En todo el proceso de pruebas, pudimos observar lo útil de un proyecto de pruebas unitarias, y estamos de acuerdo en que, junto con la depuración, nos dan detalles y una resolución acerca de cómo está funcionando nuestro proyecto y contexto. Como también es garantía para verificar que exista una relación entre la base de datos física y nuestro contexto.

\end{itemize}
